% Options for packages loaded elsewhere
\PassOptionsToPackage{unicode}{hyperref}
\PassOptionsToPackage{hyphens}{url}
%
\documentclass[
]{article}
\usepackage{lmodern}
\usepackage{amssymb,amsmath}
\usepackage{ifxetex,ifluatex}
\ifnum 0\ifxetex 1\fi\ifluatex 1\fi=0 % if pdftex
  \usepackage[T1]{fontenc}
  \usepackage[utf8]{inputenc}
  \usepackage{textcomp} % provide euro and other symbols
\else % if luatex or xetex
  \usepackage{unicode-math}
  \defaultfontfeatures{Scale=MatchLowercase}
  \defaultfontfeatures[\rmfamily]{Ligatures=TeX,Scale=1}
\fi
% Use upquote if available, for straight quotes in verbatim environments
\IfFileExists{upquote.sty}{\usepackage{upquote}}{}
\IfFileExists{microtype.sty}{% use microtype if available
  \usepackage[]{microtype}
  \UseMicrotypeSet[protrusion]{basicmath} % disable protrusion for tt fonts
}{}
\makeatletter
\@ifundefined{KOMAClassName}{% if non-KOMA class
  \IfFileExists{parskip.sty}{%
    \usepackage{parskip}
  }{% else
    \setlength{\parindent}{0pt}
    \setlength{\parskip}{6pt plus 2pt minus 1pt}}
}{% if KOMA class
  \KOMAoptions{parskip=half}}
\makeatother
\usepackage{xcolor}
\IfFileExists{xurl.sty}{\usepackage{xurl}}{} % add URL line breaks if available
\IfFileExists{bookmark.sty}{\usepackage{bookmark}}{\usepackage{hyperref}}
\hypersetup{
  hidelinks,
  pdfcreator={LaTeX via pandoc}}
\urlstyle{same} % disable monospaced font for URLs
\usepackage{color}
\usepackage{fancyvrb}
\newcommand{\VerbBar}{|}
\newcommand{\VERB}{\Verb[commandchars=\\\{\}]}
\DefineVerbatimEnvironment{Highlighting}{Verbatim}{commandchars=\\\{\}}
% Add ',fontsize=\small' for more characters per line
\newenvironment{Shaded}{}{}
\newcommand{\AlertTok}[1]{\textcolor[rgb]{1.00,0.00,0.00}{\textbf{#1}}}
\newcommand{\AnnotationTok}[1]{\textcolor[rgb]{0.38,0.63,0.69}{\textbf{\textit{#1}}}}
\newcommand{\AttributeTok}[1]{\textcolor[rgb]{0.49,0.56,0.16}{#1}}
\newcommand{\BaseNTok}[1]{\textcolor[rgb]{0.25,0.63,0.44}{#1}}
\newcommand{\BuiltInTok}[1]{#1}
\newcommand{\CharTok}[1]{\textcolor[rgb]{0.25,0.44,0.63}{#1}}
\newcommand{\CommentTok}[1]{\textcolor[rgb]{0.38,0.63,0.69}{\textit{#1}}}
\newcommand{\CommentVarTok}[1]{\textcolor[rgb]{0.38,0.63,0.69}{\textbf{\textit{#1}}}}
\newcommand{\ConstantTok}[1]{\textcolor[rgb]{0.53,0.00,0.00}{#1}}
\newcommand{\ControlFlowTok}[1]{\textcolor[rgb]{0.00,0.44,0.13}{\textbf{#1}}}
\newcommand{\DataTypeTok}[1]{\textcolor[rgb]{0.56,0.13,0.00}{#1}}
\newcommand{\DecValTok}[1]{\textcolor[rgb]{0.25,0.63,0.44}{#1}}
\newcommand{\DocumentationTok}[1]{\textcolor[rgb]{0.73,0.13,0.13}{\textit{#1}}}
\newcommand{\ErrorTok}[1]{\textcolor[rgb]{1.00,0.00,0.00}{\textbf{#1}}}
\newcommand{\ExtensionTok}[1]{#1}
\newcommand{\FloatTok}[1]{\textcolor[rgb]{0.25,0.63,0.44}{#1}}
\newcommand{\FunctionTok}[1]{\textcolor[rgb]{0.02,0.16,0.49}{#1}}
\newcommand{\ImportTok}[1]{#1}
\newcommand{\InformationTok}[1]{\textcolor[rgb]{0.38,0.63,0.69}{\textbf{\textit{#1}}}}
\newcommand{\KeywordTok}[1]{\textcolor[rgb]{0.00,0.44,0.13}{\textbf{#1}}}
\newcommand{\NormalTok}[1]{#1}
\newcommand{\OperatorTok}[1]{\textcolor[rgb]{0.40,0.40,0.40}{#1}}
\newcommand{\OtherTok}[1]{\textcolor[rgb]{0.00,0.44,0.13}{#1}}
\newcommand{\PreprocessorTok}[1]{\textcolor[rgb]{0.74,0.48,0.00}{#1}}
\newcommand{\RegionMarkerTok}[1]{#1}
\newcommand{\SpecialCharTok}[1]{\textcolor[rgb]{0.25,0.44,0.63}{#1}}
\newcommand{\SpecialStringTok}[1]{\textcolor[rgb]{0.73,0.40,0.53}{#1}}
\newcommand{\StringTok}[1]{\textcolor[rgb]{0.25,0.44,0.63}{#1}}
\newcommand{\VariableTok}[1]{\textcolor[rgb]{0.10,0.09,0.49}{#1}}
\newcommand{\VerbatimStringTok}[1]{\textcolor[rgb]{0.25,0.44,0.63}{#1}}
\newcommand{\WarningTok}[1]{\textcolor[rgb]{0.38,0.63,0.69}{\textbf{\textit{#1}}}}
\usepackage{graphicx,grffile}
\makeatletter
\def\maxwidth{\ifdim\Gin@nat@width>\linewidth\linewidth\else\Gin@nat@width\fi}
\def\maxheight{\ifdim\Gin@nat@height>\textheight\textheight\else\Gin@nat@height\fi}
\makeatother
% Scale images if necessary, so that they will not overflow the page
% margins by default, and it is still possible to overwrite the defaults
% using explicit options in \includegraphics[width, height, ...]{}
\setkeys{Gin}{width=\maxwidth,height=\maxheight,keepaspectratio}
% Set default figure placement to htbp
\makeatletter
\def\fps@figure{htbp}
\makeatother
\setlength{\emergencystretch}{3em} % prevent overfull lines
\providecommand{\tightlist}{%
  \setlength{\itemsep}{0pt}\setlength{\parskip}{0pt}}
\setcounter{secnumdepth}{-\maxdimen} % remove section numbering

\date{}

\begin{document}

\hypertarget{header-n133}{%
\section{Stochastic Heat Equation in 2D}\label{header-n133}}

\hypertarget{header-n2}{%
\subsection{I. Governing equation}\label{header-n2}}

We solve a stochastic heat equation in 2D:

\begin{align}
u_{t}&= \Delta u+\sigma \zeta(t,\mathbf{x}), \\
u|_{t=0} &= b e^{c |\mathbf{x}|^{3}} 1_{\{|\mathbf{x}| \leq r\}}. \\
\end{align}

\begin{itemize}
\item
  \(\zeta(t,\mathbf{x})=\dot{W}^{Q}(t,\mathbf{x}) =dW^{Q}(t, \mathbf{x}) /dt.\)
\item
  Therefore, we may also write:

  \begin{equation}
  du=(u_{xx}+u_{yy}) dt+\sigma dW^Q. \label{spde}
  \end{equation}
\item
  \(\zeta(t,\mathbf{x})\) is a \emph{space-time colored noise} (white in
  time, colored in space):

  \begin{itemize}
  \item
    \(\displaystyle \dot{W}^{Q}(t, \mathbf{x})= \sum_{j\ge1}\sqrt{q_j} \phi_j(\mathbf{x})\dot{w_j}(t).\)

    \begin{itemize}
    \item
      \(q_j>0\) and \(\phi_j(\mathbf{x})\) are eigenpairs of \(Q\).
    \item
      \(Q\) is the covariance operator defined by

      \(\displaystyle (Q\phi)(\mathbf{x})=\int_{D}q(\mathbf{x},\mathbf{x}^{\prime})\phi(\mathbf{x}^{\prime})d\mathbf{x}^{\prime}, \mathbf{x}\in D,\)

      with the covariance function \(q(\mathbf{x},\mathbf{x}^{\prime})\)
      as its kernel.
    \item
      \(\dot{w_j}(t) \text{ is the Gaussian white noise.}\)
    \end{itemize}
  \item
    \(\mathbb{E}\left[\zeta(t,\mathbf{x}) \zeta(s,\mathbf{x}^{\prime})\right]=\mathbb{E}\left[\dot{W}^{Q}(t, \mathbf{x}) \dot{W}^{Q}(s,\mathbf{x}^{\prime})\right]=\delta(t-s) q(\mathbf{x},\mathbf{x}^{\prime}).\)
  \item
    \(q(\mathbf{x},\mathbf{x}^{\prime})\) is the covariance function for
    its spatial correlation structure; i.e.,
    \(q(\mathbf{x},\mathbf{x}^{\prime})=Cov[\zeta(t,\mathbf{x}),\zeta(t,\mathbf{x}^{\prime})]=\mathbb{E}\left[\zeta(t,\mathbf{x}) \zeta(t,\mathbf{x}^{\prime})\right], \forall t \in [0,\infty)\).
  \end{itemize}
\item
  Since the Gaussian white noise \(\dot{w_j}(t)\) can be thought of as
  the derivative of a Brownian motion/Wiener process \(w_j(t)\), we may
  write:

  \[\displaystyle W^{Q}(t, \mathbf{x})= \sum_{j\ge1}\sqrt{q_j} \phi_j(\mathbf{x})w_j(t). \label{Qwiener}\]

  \begin{itemize}
  \item
    \( w_j(t)\) are i.i.d. Wiener process (Brownian motion). 
  \item
    It is in the form of \emph{Karhunen-Loéve (KL) expansion}.
  \end{itemize}
\item
  Here we consider:

  \begin{itemize}
  \item
    \(\mathbf{x}\in D=[0,1]^2, t\in[0,T]\). 
  \item
    \(q(\mathbf{x},\mathbf{y})\) is chosen from the covariance functions
    below:

    \begin{itemize}
    \item
      \(\exp\left(-\frac{|x_1-x^{\prime}_1|}{L_1}\right) \exp\left(-\frac{|x_2-x^{\prime}_2|}{L_2}\right)\)
      : Separable exponential
    \item
      \(\exp \left(\frac{-\left\|\mathbf{x}-\mathbf{x}^{\prime}\right\|}{L_{c}}\right)\)
      : Exponential
    \item
      \(\exp \left(\frac{-\left\|\mathbf{x}-\mathbf{x}^{\prime}\right\|^{2}}{2 L_{c}^{2}}\right)\)
      : Gaussian
    \end{itemize}
  \end{itemize}
\end{itemize}

\hypertarget{header-n63}{%
\subsection{II. Discretization}\label{header-n63}}

To discretize \(\eqref{spde}\), we apply Crank-Nicolson scheme:

\[U^{n+1}_{i,j}-U^n_{i,j}=\frac{\Delta t}{2}\left(\frac{1}{\Delta x^2}(U^{n+1}_{i+1,j}-2U^{n+1}_{i,j}+U^{n+1}_{i-1,j})\\+\frac{1}{\Delta y^2}(U^{n+1}_{i,j+1}-2U^{n+1}_{i,j}+U^{n+1}_{i,j-1})+
\frac{1}{\Delta x^2}(U^{n}_{i+1,j}-2U^{n}_{i,j}+U^{n}_{i-1,j})+\\\frac{1}{\Delta y^2}(U^{n}_{i,j+1}-2U^{n}_{i,j}+U^{n}_{i,j-1})\right)+\sigma \left((W^Q)^{n+1}-(W^Q)^{n}\right).\]

Since \(w_j(t_{n+1})-w_j(t_n)\sim N(0, \Delta t)\), rewrite the
stochastic part as:

\[(W^Q)^{n+1}-(W^Q)^{n}=\sqrt{\Delta t}\sum_{j\ge1}\sqrt{q_j}\phi_j(\mathbf{x})\xi^n_{j}.\]

, where \(\xi^n_j:=(w_j(t_{n+1})-w_j(t_n))/\sqrt{\Delta t}\sim N(0,1)\)
i.i.d. are easily sampled.

\hypertarget{header-n139}{%
\subsection{III. Computation of KL expansion}\label{header-n139}}

Given \(q(\mathbf{x},\mathbf{x}^\prime)\), we need to find eigenpairs
\{\(q_{j},\phi_j\)\} of \(Q\) by solving:

\[\int_D q(\mathbf{x},\mathbf{x}^\prime)\phi_j(\mathbf{x^\prime})d\mathbf{x}^\prime=q_j\phi_j(\mathbf{x}).\]

It is called Fredholm integral equation of the second kind.

\hypertarget{header-n223}{%
\subsubsection{A. Approximation via collocation and
quadrature}\label{header-n223}}

\hypertarget{header-n154}{%
\paragraph{\texorpdfstring{ 1. Collocation
method}{ 1. Collocation method}}\label{header-n154}}

Let \(\mathbf{x}_1,\mathbf{x}_2,…,\mathbf{x}_{P}\) be points in \(
D\), define

\[R_j:=\int_D q(\mathbf{x},\mathbf{x}^\prime)\phi_j(\mathbf{x}^\prime)d\mathbf{x}^\prime-q_j\phi_j(\mathbf{x}).\]

If \(R_j(x_k)=0\) for \(k=1,…,P\), \{\(q_j,\phi_j\)\} is called a
collocation approximation.

\hypertarget{header-n170}{%
\paragraph{\texorpdfstring{ 2.
Quadrature}{ 2. Quadrature}}\label{header-n170}}

Suppose the \(\mathbf{x}_k\) are quadrature points with weights
\(\varpi_k\) chosen such that:

\[\int_D q(\mathbf{x}_k,\mathbf{x^\prime})\phi(\mathbf{x^\prime})d\mathbf{x^\prime}\approx \sum^{P}_{i=1}\varpi_iq(\mathbf{x}_k,\mathbf{x_i})\phi(\mathbf{x_i}).\]

\hypertarget{header-n163}{%
\subsubsection{\texorpdfstring{ B. Eigenvalue
problem}{ B. Eigenvalue problem}}\label{header-n163}}

Thus, we obtain the collocation approximation by solving:

\[\sum^{P}_{i=1}\varpi_i q(\mathbf{x_k},\mathbf{x_i})\phi_j(\mathbf{x_i})=q_j\phi_j(\mathbf{x_k}), \ k=1,...,P.\]

In matrix notation:

\[CW\psi_j=q_j\psi_j\]

, where \(C\in \mathbb{R}^{P\times P}\) is the covariance matrix for the
quadrature points, \(W\in \mathbb{R}^{P\times P}\) is the diagonal
matrix of weights and \(\psi_j\) contains point evaluations of
\(\phi_j\).

This is an eigenvalue problem for matrix \(CW\), which may not be
symmetric. We may alternatively consider the symmetric matrix
\(K=W^{1/2}CW^{1/2}\), since then

\[Kz_j=q_jz_j \label{symeig}\]

with \(z_j:=W^{1/2}\psi_j\).

\hypertarget{header-n207}{%
\subsubsection{C. SVD decomposition}\label{header-n207}}

We may apply SVD decomposition to solve \(\eqref{symeig}\):

\[K=USV'\]

, since \(K\) is a symmetric positive definite. (Recall \(q_j>0\) is
assumed \(\forall j\in \mathbb{N}\).)

Here \(S\) contains the eigenvalues \(q_j\) in non-increasing order, and
\(U=V\) contains all the corresponding eigenvectors \(z_j\) of \(K\).
The point evaluations \(\psi_j\) of the eigenvectors \(\phi_j\) can be
recovered by \(\psi_j=W^{-1/2}z_j\) .

\hypertarget{header-n220}{%
\subsubsection{D. Vertex-based quadrature}\label{header-n220}}

Partition \(D=[0,1]^2\) into \((N_x-1)\times (N_y-1)\) rectangles with
edge with side length \(h_x=1/(N_x-1)\), \(h_y=1/(N_y-1)\) and let
\(\mathbf{x_k}, k=1,2,…,P=N_xN_y\), be the vertices. We apply the
quadrature (trapezoidal) rule with weights:

\[w_{i}=\left\{\begin{array}{ll}{h_xh_y/ 4,} & {\text { if } x_{i} \text { is a corner of } D,} \\ {h_xh_y / 2,} & {\text { if } x_{i} \text { lies on an edge of } D,} \\ {h_xh_y,} & {\text { if } x_{i} \text { lies in the interior of } D.}\end{array}\right.\]

\hypertarget{header-n225}{%
\subsubsection{E. Truncated KL expansion}\label{header-n225}}

Consider the finite sum approximation of \(W^Q\) in \(\eqref{Qwiener}\):

\[W_J=\sum^{N_{KL}}_{j=1}\sqrt{q_j}\phi_j(\mathbf{x})w_{j}(t).\]

Here we keep all terms in KL expansion \(N_{KL}=N_xN_y\).

\hypertarget{header-n131}{%
\subsection{IV. Test}\label{header-n131}}

\hypertarget{header-n316}{%
\subsubsection{\texorpdfstring{A. Gaussian covariance function:
\(q(\mathbf{x},\mathbf{x}^{\prime})=\exp \left(\frac{-\left\|\mathbf{x}-\mathbf{x}^{\prime}\right\|^{2}}{2 L_{c}^{2}}\right)\)}{A. Gaussian covariance function: q(\textbackslash mathbf\{x\},\textbackslash mathbf\{x\}\^{}\{\textbackslash prime\})=\textbackslash exp \textbackslash left(\textbackslash frac\{-\textbackslash left\textbackslash\textbar\textbackslash mathbf\{x\}-\textbackslash mathbf\{x\}\^{}\{\textbackslash prime\}\textbackslash right\textbackslash\textbar\^{}\{2\}\}\{2 L\_\{c\}\^{}\{2\}\}\textbackslash right)}}\label{header-n316}}

\begin{Shaded}
\begin{Highlighting}[]
\NormalTok{>> ./ex13_SDE -matlab-engine-graphics -da_refine }\FloatTok{1}
\end{Highlighting}
\end{Shaded}

\begin{itemize}
\item
  \(N_x=N_y=31, \Delta t=0.01, t_{final}=0.20,\\b=5,c=-30,r=0.5,\\\sigma=1.5,L_{c}=2\)

  \begin{figure}
  \centering
  \includegraphics{https://bitbucket.org/petsc/petsc/raw/7cba5aba8fd1ead1681cfd3836ae6948e0fbc327/src/ts/examples/tutorials/Gaussian_sigma\%3D1.5\%2CLc\%3D2.gif}
  \caption{}
  \end{figure}

  https://bitbucket.org/petsc/petsc/raw/7cba5aba8fd1ead1681cfd3836ae6948e0fbc327/src/ts/examples/tutorials/Gaussian\_sigma\%3D1.5\%2CLc\%3D2.gif
\end{itemize}

\hypertarget{header-n414}{%
\subsubsection{\texorpdfstring{B. Separable exponential covariance
function:
\(q(\mathbf{x},\mathbf{x}^{\prime})=\exp\left(-\frac{|x_1-x^{\prime}_1|}{L_1}\right) \exp\left(-\frac{|x_2-x^{\prime}_2|}{L_2}\right)\)}{B. Separable exponential covariance function: q(\textbackslash mathbf\{x\},\textbackslash mathbf\{x\}\^{}\{\textbackslash prime\})=\textbackslash exp\textbackslash left(-\textbackslash frac\{\textbar x\_1-x\^{}\{\textbackslash prime\}\_1\textbar\}\{L\_1\}\textbackslash right) \textbackslash exp\textbackslash left(-\textbackslash frac\{\textbar x\_2-x\^{}\{\textbackslash prime\}\_2\textbar\}\{L\_2\}\textbackslash right)}}\label{header-n414}}

\begin{Shaded}
\begin{Highlighting}[]
\NormalTok{>> ./ex13_SDE -matlab-engine-graphics}
\end{Highlighting}
\end{Shaded}

\begin{itemize}
\item
  \(N_x=N_y=16, \Delta t=0.01, t_{final}=0.20,\\b=5,c=-30,r=0.5,\\\sigma=1.5,L_{x}=0.1,L_{y}=0.1\)
\end{itemize}

With the warning message from svd():

Warning: Reached maximum number of sweeps (64) in SVD routine

\begin{figure}
\centering
\includegraphics{https://bitbucket.org/petsc/petsc/raw/7cba5aba8fd1ead1681cfd3836ae6948e0fbc327/src/ts/examples/tutorials/SepExp_sigma\%3D1.5\%2CLx\%3D0.1\%2CLy\%3D0.1.gif}
\caption{}
\end{figure}

https://bitbucket.org/petsc/petsc/raw/7cba5aba8fd1ead1681cfd3836ae6948e0fbc327/src/ts/examples/tutorials/SepExp\_sigma\%3D1.5\%2CLx\%3D0.1\%2CLy\%3D0.1.gif

\hypertarget{header-n1068}{%
\subsubsection{C. Exponential covariance function:}\label{header-n1068}}

\hypertarget{header-n439}{%
\subsubsection{\texorpdfstring{\(q(\mathbf{x},\mathbf{x}^{\prime})=\exp\left(-\frac{|x_1-x^{\prime}_1|}{L_1}\right) \exp\left(-\frac{|x_2-x^{\prime}_2|}{L_2}\right)\)}{q(\textbackslash mathbf\{x\},\textbackslash mathbf\{x\}\^{}\{\textbackslash prime\})=\textbackslash exp\textbackslash left(-\textbackslash frac\{\textbar x\_1-x\^{}\{\textbackslash prime\}\_1\textbar\}\{L\_1\}\textbackslash right) \textbackslash exp\textbackslash left(-\textbackslash frac\{\textbar x\_2-x\^{}\{\textbackslash prime\}\_2\textbar\}\{L\_2\}\textbackslash right)}}\label{header-n439}}

\begin{Shaded}
\begin{Highlighting}[]
\NormalTok{>> ./ex13_SDE -matlab-engine-graphics}
\end{Highlighting}
\end{Shaded}

\begin{itemize}
\item
  \(N_x=N_y=16, \Delta t=0.01, t_{final}=0.20,\\b=5,c=-30,r=0.5,\\\sigma=1.5,L_{x}=0.1,L_{y}=0.1\)
\end{itemize}

With the warning message from svd():

Warning: Reached maximum number of sweeps (64) in SVD routine

\begin{figure}
\centering
\includegraphics{https://bytebucket.org/petsc/petsc/raw/7cba5aba8fd1ead1681cfd3836ae6948e0fbc327/src/ts/examples/tutorials/Exp_sigma\%3D1.5\%2CLc\%3D2.gif}
\caption{}
\end{figure}

https://bytebucket.org/petsc/petsc/raw/7cba5aba8fd1ead1681cfd3836ae6948e0fbc327/src/ts/examples/tutorials/Exp\_sigma\%3D1.5\%2CLc\%3D2.gif

\end{document}
